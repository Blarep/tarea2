\documentclass[spanish, fleqn]{article}
\usepackage{babel}
\usepackage[utf8]{inputenc}
\usepackage{amsmath,amsfonts}
\usepackage{enumitem}
\usepackage[colorlinks, urlcolor=blue]{hyperref}
\usepackage{fourier}
\usepackage[top = 2.5cm, bottom = 2cm, left = 2.5cm, right = 2.5cm]{geometry}

\newcommand{\num}{2}

\title{Estructuras Discretas \\
       Tarea \#\num \\
       ``Demuestre que sabe demostrar''}
\author{Discrete Structure Warriors}
\date{}

\begin{document}
\maketitle
\thispagestyle{empty}

% Pregunta 1 Laura
\section*{Pregunta 1}
Demuestre por inducción sobre \(n\) que
\[
\prod_{2 \leq k \leq n} \left( 1 - \frac{1}{k^2} \right) = \frac{n+1}{2n}
\]
\hfill (30 ptos.)

%% Pregunta 2 Laura
\section*{Pregunta 2}
Para vectores \( \vec{x} = (x_1, x_2, x_3, \cdots, x_n) \) se define la norma Euclidiana y la norma Infinita de la siguiente forma:
\begin{align*}
	|| \vec{x} ||_2 &= 
    \left(
    \sum_{i=1}^{n} | x_i |^2
    \right)^\frac{1}{2}  = \sqrt{|x_1|^2+|x_2|^2+ ... + |x_n|^2}\\
    &\\
    || \vec{x} ||_{\infty} &= 
    \max_{i=1}^{n} |x_i| =  \max(|x_1|,|x_2|,...,|x_n|) 
\end{align*}

Considere \(n = 2\). Para cualquier vector \( \vec{x} \), use \textbf{deducción} para mostrar que:

\[
|| \vec{x} ||_2 \leq \sqrt{n} \cdot  || \vec{x} ||_{\infty}
\]

\textbf{Pista:} Puede serle útil usar la variable auxiliar \(x_m = \max(|x_1|, |x_2|)\).

\hfill (40 ptos.)

%% Pregunta 3
\section*{Pregunta 3}
\begin{enumerate}
\item Demuestre por contradicción que si una función \(f: \mathcal{R} \rightarrow \mathcal{R}\) estrictamente creciente entonces es inyectiva.\\

\item Elija otro método  (que no sea contradicción) para demostrar lo anterior.
\end{enumerate}


\hfill (30 ptos.)


  \vfill\hfill DSW/\LaTeXe
\end{document}
