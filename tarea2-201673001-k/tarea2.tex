\documentclass[spanish, fleqn]{article}
\usepackage[spanish]{babel}
\usepackage[utf8]{inputenc}
\usepackage{amsmath,amsfonts,amssymb,mathtools}
\usepackage{enumitem}
\usepackage[colorlinks, urlcolor=blue]{hyperref}
\usepackage{listings}
\usepackage{verbatim}
\usepackage[top = 2.5cm, bottom = 2cm, left = 2.5cm, right = 2.5cm]{geometry}
\usepackage{cancel}
\setlength{\parskip}{4mm}
\pagestyle{empty}

\newcommand{\num}{2}

\title{Estructuras Discretas \\
       Tarea \#\num \\
       ``Demuestre que sabe demostrar''}
\author{Andrés Navarro \\ (201673001-k)}
\date{}

\begin{document}
\maketitle
\thispagestyle{empty}

\section*{Pregunta 1}
Demuestre por inducción sobre \(n\) que
\[
\prod_{2 \leq k \leq n} \left( 1 - \frac{1}{k^2} \right) = \frac{n+1}{2n}
\]
\begin{itemize}
\item Para $n=2$:
\begin{align*}
\hspace{-0.7cm}\prod_{2 \leq k \leq n} \left( 1 - \frac{1}{k^2} \right)=\left( 1 - \frac{1}{2^2} \right)=\left( 1 - \frac{1}{4} \right)=\left(\frac{4-1}{4} \right)=\left( \frac{3}{4} \right)
\end{align*}

Dado que:
\begin{align*}
\hspace{-0.7cm}\prod_{2 \leq k \leq n} \left( 1 - \frac{1}{k^2} \right) = \frac{n+1}{2n} = \frac{2+1}{2 \cdot 2} = \frac{3}{4}
\end{align*}
$\therefore $ Si se cumple el caso base.
\item Suponemos que para $n=m$ esto es verdadero:
\begin{align*}
\hspace{-0.7cm}\prod_{2 \leq k \leq m} \left( 1 - \frac{1}{k^2} \right) = \frac{m+1}{2m}
\end{align*}
\item Entonces, para $n=m+1$:
\begin{flalign*}
\prod_{2 \leq k \leq m+1} \left( 1 - \frac{1}{k^2} \right) &= \prod_{2 \leq k \leq m} \left( 1 - \frac{1}{k^2} \right) \cdot \left( 1 - \frac{1}{(m+1)^2} \right)&\\
&=\frac{m+1}{2m} \cdot \left( 1 - \frac{1}{(m+1)^2} \right)&\\
&=\frac{\cancel{m+1}}{2m} \cdot \left(\frac{(m+1)^2-1}{(m+1)^{\cancel{2}}} \right)&\\
&=\frac{m^2+2m+\cancel{1}-\cancel{1}}{2m \cdot (m+1)}&\\
&=\frac{\cancel{m}\cdot(m+2)}{2\cancel{m} \cdot (m+1)}&\\
&=\frac{m+2}{2m+2}&
\end{flalign*}
Esto es precisamente la proposición para n=m+1:
\begin{align*}
\hspace{-0.7cm}\prod_{2 \leq k \leq m+1} \left( 1 - \frac{1}{k^2} \right)  = \frac{n+1}{2n}=\frac{(m+1)+1}{2(m+1)} = \frac{m+2}{2m+2}
\end{align*}
\end{itemize}

$\therefore$ Queda demostrado por inducción que:
\[
\prod_{2 \leq k \leq n} \left( 1 - \frac{1}{k^2} \right) = \frac{n+1}{2n}
\]


\section*{Pregunta 2}
Para vectores \( \vec{x} = (x_1, x_2, x_3, \cdots, x_n) \) se define la norma Euclidiana y la norma Infinita de la siguiente forma:
\begin{align*}
	|| \vec{x} ||_2 &= 
    \left(
    \sum_{i=1}^{n} | x_i |^2
    \right)^\frac{1}{2}  = \sqrt{|x_1|^2+|x_2|^2+ ... + |x_n|^2}\\
    &\\
    || \vec{x} ||_{\infty} &= 
    \max_{i=1}^{n} |x_i| =  \max(|x_1|,|x_2|,...,|x_n|) 
\end{align*}

Considere \(n = 2\). Para cualquier vector \( \vec{x} \), use \textbf{deducción} para mostrar que:
$|| \vec{x} ||_2 \leq \sqrt{n} \cdot  || \vec{x} ||_{\infty}$.

\textbf{Pista:} Puede serle útil usar la variable auxiliar \(x_m = \max(|x_1|, |x_2|)\).

Para n=2 tenemos:
\begin{align*}
	&|| \vec{x} ||_2 = \left( \sum_{i=1}^{2} | x_i |^2 \right)^\frac{1}{2}  = \sqrt{|x_1|^2+|x_2|^2}\\
    &|| \vec{x} ||_{\infty} =  \max_{i=1}^{2} |x_i| =  \max(|x_1|,|x_2|)=x_m 
\end{align*}

Luego:
\begin{align*}
|| \vec{x} ||_2 \leq \sqrt{n} \cdot  || \vec{x} ||_{\infty}\Rightarrow \sqrt{|x_1|^2+|x_2|^2}&\leq\sqrt{2}\cdot x_m&\\
\sqrt{|x_1|^2+|x_2|^2}&\leq\sqrt{2}\cdot x_m \hspace{1cm}/()^2&\\
|x_1|^2+|x_2|^2&\leq2\cdot (x_m)^2&
\end{align*}

En el caso de que $x_1$ sea el máximo ($x_m$):
\begin{align*}
|x_1|^2+|x_2|^2&\leq2\cdot |x_1|^2&\\
|x_2|^2&\leq|x_1|^2 \hspace{1cm}/()^{1/2}&\\
|x_2| &\leq|x_1|
\end{align*}

Lo que es cierto ya que $|x_1|$ es el máximo entre $|x_1|$ y $|x_2|$

Análogamente, si $x_2$ es el máximo, llegamos a otra desigualdad que también es verdadera:
\begin{align*}
|x_1|^2+|x_2|^2&\leq2\cdot |x_2|^2&\\
|x_1|^2&\leq|x_2|^2 \hspace{1cm}/()^{1/2}&\\
|x_1|&\leq|x_2|
\end{align*}

$\therefore$ Para n=2 queda demostrado por deducción que:

\hspace{0.4cm}$|| \vec{x} ||_2 \leq \sqrt{n} \cdot  || \vec{x} ||_{\infty}$

\section*{Pregunta 3}
\begin{enumerate}
\item Demuestre por contradicción que si una función \(f: \mathcal{R} \rightarrow \mathcal{R}\) estrictamente creciente entonces es inyectiva.

Sea $x_1,x_2\in Dom(f)$.

Se tiene que:
\begin{align*}
f \textit{es estrictamente creciente} &\Rightarrow f \textit{es inyectiva}\\
\underbrace{(x_1 < x_2 \Rightarrow f(x_1)< f(x_2))}_{p}&\Rightarrow  \underbrace{(\underbrace{x_1\neq x_2}_{r} \Rightarrow \underbrace{f(x_1)\neq f(x_2)}_{s})}_{q}\\
p &\Rightarrow q\\
&\Downarrow \textit{negando la implicancia}\\
\neg (p&\Rightarrow q)\\
p &\wedge \neg q \\
p &\wedge \neg (r \Rightarrow s)\\
p &\wedge r \wedge \neg s\\
&\Downarrow \textit{reemplazando las proposiciones}\\
(x_1 < x_2 \Rightarrow f(x_1)< f(x_2))&\wedge x_1\neq x_2 \wedge f(x_1)= f(x_2)
\end{align*}

Luego de obtener esto, se llega a un absurdo, dado que para dos números, $x_1$ y $x_2$, siendo $x_1$ menor (y obviamente distinto) que $x_2$, se debería cumplir que $f(x_1)$ sea igual a $f(x_2)$ y que simultáneamente $f(x_1)$ sea menor a $f(x_2)$, situación que no puede ocurrir.

$\therefore$ Queda demostrado por contradicción que:

$\hspace{0.4cm} f \textit{es estrictamente creciente} \Rightarrow f \textit{es inyectiva}$
 
\item Elija otro método  (que no sea contradicción) para demostrar lo anterior.

Por deducción:

Sea $x_1,x_2\in Dom(f)$.

Suponemos que $x_1<x_2$.

Entonces, se tiene que:
\begin{align*}
f \textit{es estrictamente creciente} &\Rightarrow f \textit{es inyectiva}\\
\underbrace{(x_1 < x_2 \Rightarrow f(x_1)< f(x_2))}_{p}&\Rightarrow \underbrace{(x_1\neq x_2\Rightarrow f(x_1)\neq f(x_2))}_{q}\\
\end{align*}
Dada la informacion de $p$ (condiciones para que la función sea estrictamente creciente), podemos decir lo siguiente:

-Si $x_1<x_2$, se tiene que $x_1 \neq x_2$.

-Si $f(x_1)<f(x_2)$, se tiene que $f(x_1) \neq f(x_2)$.

Con esto, se tiene que $q$ es consecuencia lógica de $p$.

$\therefore$ Queda demostrado por deducción que:

$\hspace{0.4cm} f \textit{es estrictamente creciente} \Rightarrow f \textit{es inyectiva}$
\end{enumerate}

\end{document}
